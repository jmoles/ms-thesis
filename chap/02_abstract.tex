Chemical reaction networks are an unconventional computing medium that could benefit from the ability to form basic control systems. In this work, we demonstrate the functionality of a chemical control system by evaluating classic genetic algorithm problems: Koza's Santa Fe trail, Jefferson's John Muir trail, and three Santa Fe trail segments. Both Jefferson and Koza found that memory, such as a recurrent neural network or memories in a genetic program, are required to solve the task. Our approach presents the first instance of a chemical system acting as a control system. We propose a delay line connected with an artificial neural network in a chemical reaction network to determine the artificial ant's moves. 

We first search for the minimal required delay line size connected to a feed forward neural network in a chemical system. Our experiments show a delay line of length four is sufficient. Next, we used these findings to implement a chemical reaction network with a length four delay line and an artificial neural network. We use genetic algorithms to find an optimal set of weights for the artificial neural network. This chemical system is capable of consuming 100\% of the food on a subset and greater than 44\% of the food on Koza's Santa Fe trail.

We also show the first implementation of a simulated chemical memory in two different models that can reliably capture and store information over time. The ability to store data over time gives rise to basic control systems that can perform more complex tasks. The integration of a memory storage unit and a control system in a chemistry has applications in biomedicine, like smart drug delivery. We show that we can successfully store the information over time and use it to act as a memory for a control system navigating an agent through a maze. 